\documentclass[a4paper]{article}

%% Language and font encodings
\usepackage[english]{babel}
\usepackage[utf8x]{inputenc}
\usepackage[T1]{fontenc}

%% Sets page size and margins
\usepackage[a4paper,top=3cm,bottom=2cm,left=3cm,right=3cm,marginparwidth=1.75cm]{geometry}

%% Useful packages
\usepackage{amsmath}
\usepackage{amsfonts}
\usepackage{commath}
\usepackage{graphicx}
\usepackage{tikz-cd}
\usepackage{bm}
\usepackage[colorinlistoftodos]{todonotes}
\usepackage[colorlinks=true, allcolors=blue]{hyperref}

%% Other character defs
\newcommand*\diff{\mathop{}\!\mathrm{d}}
\renewcommand\vec{\mathbf}	% Using '\vec{}` is equivalent to \mathbf

\title{Project Proposal: Comparing Optimization and Estimation Techniques for Low-Thrust Spacecraft Rendezvous}
\author{Andrew Cox, Mike Sparapany, Collin York, Waqar Zaidi \\\textit{Purdue University}}
\date{}	% Don't include a date in the title

\begin{document}
\maketitle

% \begin{abstract}
% Low-thrust technologies offer efficient mass usage which can lower spacecraft mass and, thus, mission costs. However, due to the small forces generated by low-thrust engines, long-duration burns are required to significantly alter a spacecraft path. Accordingly, low-thrust trajectory design must incorporate additional variables to describe the thrust vector during the burn duration. The solution to this higher-dimensional trajectory design problem is often obtained via optimal control where the thrust vector constitutes the set of control variables. In this study, a rendezvous scenario between a low-thrust-equipped spacecraft and an uncontrolled ``mothership'' is considered. Uncertainties in the spacecraft position, velocity, and thrust vectors are mitigated via estimation algorithms and a mass-optimal trajectory is obtained that satisfies the natural dynamics and mission-imposed constraints. Several techniques for optimization and estimation are investigated and their effects on the optimal solution are discussed.
% \end{abstract}

\section{Introduction}
Rendezvous of a chaser spacecraft with a target spacecraft in a predetermined orbit is a common problem in spaceflight operations. Spacecraft mass is frequently minimized during such operations to reduce overall mission expenditures. Advancements in low-thrust propulsion technology facilitate these minimizations by transforming propellant mass into a propulsive force more efficiently than other technologies. However, due to the small accelerations delivered by low-thrust systems, significant burn durations are required to alter the course of the spacecraft. Accordingly, low-thrust trajectory design must incorporate variables to describe the thrust vector in addition to the usual variables that describe a spacecraft state. The following proposed analysis attempts to blend these two traditional spaceflight problems and determine optimization and estimation techniques that allow rendezvous using low-thrust propulsion systems.

\section{Problem Definition}
% A nonlinear control system, $\Sigma$, is the 4-tuple $(B, Q, \pi, \bm{f})$ consisting of a system's state-space manifold, $Q \in (\bm{x})$, and state-dependent input-space manifold, $B \in (\bm{x}, \bm{u})$, so that $\pi : B \to M$ is a smooth fiber bundle. Additionally, a user-supplied vector field is defined as $\bm{f} : B \to TQ$. Considering the natural projection of $TQ$ on $Q$, $\tau_Q : TQ \to Q$, then $\pi = \tau_Q \circ \bm{f}$. This setup is summarized in the following commutative diagram

% \begin{center}
% 	\begin{tikzcd}[column sep=large, row sep=large]
% 		& B \arrow[swap, dl, "\bm{f}"] \arrow[d, "\pi"] \\
% 		TQ \arrow[r, "\tau_Q"] & Q
% 	\end{tikzcd}
% \end{center}

% A trajectory is a curve, $\gamma(t) : I \subset \mathbb{R} \to B$, that is also an integral curve of $\bm{f}$.

Let the state vector, $\vec{x} = \{r,~ \theta,~ \dot{r},~ \dot{\theta},~ m\}$, store the spacecraft position as the radius, $r$, and angle $\theta$, relative to an inertial reference frame centered on Earth. Similarly, the velocity is described by the time derivatives $\dot{r}$ and $\dot{\theta}$. Finally, the spacecraft mass, $m$, is included as mass varies during propulsive burns. The control vector, $\vec{u} = \{T,~ \alpha \}$ includes the thrust magnitude, $T$, and the orientation of the thrust vector relative to the inertial frame, $\alpha$.

The dynamics governing the state are the well-known Keplerian equations of motion augmented with uncertainties which may include non-spherical gravity, solar radiation pressure, drag, or other forces. In equation form,
\begin{equation}
	\vec{f} = \dot{\vec{x}} = 
    \begin{Bmatrix}
    	\dot{r}\\
        \dot{\theta}\\
        r \dot{\theta} - \mu/r^2 + \frac{T}{m} \left(C_{\alpha}C_{\theta} + S_{\alpha}S_{\theta}\right)\\
        -2\dot{r}\dot{\theta}/r + \frac{T}{mr} \left( S_{\alpha} C_{\theta} - C_{\alpha}S_{\theta} \right)\\
        -T/(I_{sp} g_0)
    \end{Bmatrix} + \vec{w}\,,
\end{equation}
where $\mu$ is the standard gravity parameter for Earth, $\mu = 3.986$e05 km$^3$/s$^2$, $I_{sp}$ is the propulsion system specific impulse, $g_0$ is the mean Earth gravitational acceleration, $g_0 = 9.80665$e-03 km/s$^2$, and $\vec{w}$ is a vector of noise that encapsulates perturbations to the conic model.

The goal of the rendezvous problem is to identify an optimal control history, $\vec{u}^*$ that maximizes the spacecraft mass (i.e., minimizes the mass spent during the flight) and delivers the spacecraft to a desired state $\vec{x}_d$. The initial state, $\vec{x}(0) = \vec{x}_0$, is fixed, as is the final thrust magnitude, $T(t_f) = 0$.

% \textbf{NOTES}
% \begin{itemize}
% 	\item How does estimation work into the problem?
%     \item Is this formulation usable? We could also use cartesian states. Or, we could have $\alpha$ orient the thrust vector relative to the spacecraft velocity, or a fixed vector in the rotating frame fixed on the spacecraft.
% \end{itemize}

% \subsection{Cartesian Equations (for reference)}
% \textit{Here are the cartesian EOMs, in case we want to use the instead.}\\
% \noindent Let $\vec{x} = \{x,~ y,~ \dot{x},~ \dot{y},~ m \}$. The control vector, $\vec{u}$, remains the same as previously defined. The equations of motion are
% \begin{equation}
% 	\dot{\vec{x}} = \begin{Bmatrix}
%     	\dot{x}\\
%         \dot{y}\\
%         -(\mu/r^2) C_{\theta} + (T/m) C_{\alpha}\\
%         -(\mu/r^2) S_{\theta} + (T/m) S_{\alpha}\\
%         -T/(I_{sp} g_0)
%     \end{Bmatrix} + \vec{w} \,,
% \end{equation}
% where $\theta = \arctan(y/x)$ and $r = \sqrt{x^2 + y^2}$.

\section{Orbit Estimation}
In the context of low-thrust optimization, recursive Bayesian updates are applied for each measurement received between successive trajectory subarcs. However, instead of using the traditional Extended Kalman Filter (EKF), which simply linearizes all the nonlinear transformations in the KF equations, the capabilities of an Unscented Kalman Filter (UKF) that uses a set of discretely sampled points to parameterize the system mean and covariance are leveraged. Essentially, the UKF represents a mixture between an EKF and Monte Carlo whereby the sampled or "sigma" points are used to propagate uncertainty through complete nonlinear transformations. Accordingly, no Jacobians are required to develop a state transition matrix that linearly predicts future uncertainty prior to a sequential Bayesian update. Additionally, the UKF implementation will incorporate process noise (at each node between trajectory subarcs) and $\it resample$ sigma points to account for uncertainty due to suboptimal forces.  The state and measurement update equations for the UKF are provided here, \\

State-Update
\begin{equation}
\begin{aligned}
\bar{X}_t &= \sum_{i=0}^{2L} W^{m}_i \chi_{i,t/t-1} \\
\bar{P}_t &= Q + \sum_{i=0}^{2L} W^{c}_i (\chi_{i,t/t-1} - \bar{X}_t)(\chi_{i,t/t-1} - \bar{X}_t)^T  \\
\end{aligned}
\end{equation}

Measurement-Update
\begin{equation}
\begin{aligned}
K_t &= P_{xy}P^{-1}_{yy} \\
\hat{X}_t &= \bar{X}_t + K_t(y_t - \bar{y}_t) \\
P_t &= \bar{P}_t - K_tP_{yy}K^{T}_t \\
\end{aligned}
\end{equation}

where $\it L$ is the number of sigma points, $\it W^m$ are weights that capture distribution statistics, $\chi$ are sigma points, $\it Q$ is the process-noise matrix, $\bar{y}_t$ are simulated  measurements, and ${y}_t$ are true measurements. The mapping of the measurement space (i.e. topocentric spherical coordinates) to inertial coordinates is given by,

\begin{equation}
\begin{aligned}
\vec{r} = \vec{o} + \rho \vec{u_{\rho}} \\
\vec{\dot{r}} = \vec{\dot{o}} + \dot{\rho} \vec{u_{\rho}} + \rho \dot{\alpha} \cos\delta \vec{u_{\alpha}} + \rho \delta \vec{u_{\delta}} \\
\end{aligned}
\end{equation}

where $\vec{o}$ and $\vec{\dot{o}}$ are the ground station inertial position and velocity coordinates at measurement time.

\section{Proposed Analysis}

Given a fixed target vehicle trajectory and chaser vehicle initial conditions and covariance, the team will compare combinations of two optimization methods and two state estimation methods. Starting with an initial chaser state, a mass-optimal trajectory will be determined via either standard indirect methods or the Marsden-Weinstein-Meyer process. We expect the MWM process to be significantly faster and more efficient over standard indirect methods. The actual trajectory will be propagated forward and subjected to a stochastic process noise, simulating perturbing accelerations which are not reflected in the equations of motion. After a set time interval, a measurement will be recorded and subjected to stochastic measurement noise. The measurement will then be used to determine a state estimation and associated trajectory dispersion covariance via either an EKF or UKF. Using the state estimate, the same optimization method will be re-run to calculate a new control plan. This entire process will iterate until the chaser covariance ellipsoid overlaps with the target vehicle state and is under a given tolerance.

To compare and contrast the four total combinations of optimization and estimation, the team will run multiple trials of each set and compare mass spent and time of flight. Since the state estimate will be used to determine successful rendezvous, the team will also analyze the final error between the actual and estimated states. The analysis will be repeated for ranges of process and measurement noise to look for trends. Since the EKF propagatates mean and covariance linearly, we expect large errors due to the non-linear equations of motion. Conceptually, we expect prediction off of measurement updates using the UKF to produce miss distances between target and chaser vehicles that are inside the chaser covariance ellipsoid. If the predicted miss distance is outside the ellipsoid (i.e. due to a EKF measurement update), thrusting will be required to bring the miss inside the ellipsoid.
% \\
% \textbf{NOTES TO TEAM}\\
% Waqar and Mike - this would be a great place to type a sentence or two about what differences you expect between EKF and UKF, and ID Opt. vs. M-W-M (I threw in a sentence saying MWM is faster, maybe, idk).

% Given a nonlinear control system, $\Sigma$, and a path cost, $L : B \to \mathbb{R}$, define the optimal control problem as

% \begin{equation} \label{eq:cost}
% \min_{u} J = \min_{u} \int_{t_0}^{t_f} L \circ \gamma \diff t
% \end{equation}

% subject to:

% \begin{equation}
% \begin{aligned}
% \dod{}{t}(\pi \circ \gamma) &= \bm{f} \circ \gamma \\
% \bm{\Phi} \circ \gamma(t_0) &= 0 \\
% \bm{\Psi} \circ \gamma(t_f) &= 0
% \end{aligned}
% \end{equation}

% Indirect methods of trajectory optimization refers to introducing a Hamiltonian structure to the optimization problem formulation and searching for solutions to the infinite-dimensional problem in a finite-dimensional phase space, or cotangent bundle, $T^*Q=M$ with $\pi_Q : T^*Q \rightarrow Q$. This is done analytically by introducing Lagrange multipliers as fiber coordinates, $\bm{\lambda} \in \pi_Q^{-1}$, and augmented the cost functional $J$. The phase space, $M$, together with it's symplectic form, $\omega$, is a symplectic manifold. \\

% The Marsden-Weinstein-Meyer reduction theorem formalizes the reduction process for symplectic manifolds. The reduction process is a two-step procedure beginning with identifying symmetries, constants-of-motion, and their corresponding group structure. First, take a symplectic manifold, $(M,\omega)$, and assume a Lie group of symmetries, $G$, acts symplectically on this manifold. Define the moment map to be $\mu : M \rightarrow \mathfrak{g}^*$. Evaluate the moment map, or constants-of-motion, at a value $0 \in \mathfrak{g}^*$. Using this momentum map, restrict motion to the set of all reachable states, $\mu^{-1}(0)$. This ensures optimal motion on $(M, \omega)$ is also on level sets of $\mu$. Because motion is dependent on values of $\mu$, this implies the value of $\mu$ must be known a priori. This is not an issue for optimal control systems since an initial guess is required by the majority of algorithms, thus a potentially sub-optimal value of $\mu$ is guaranteed. \\

% The second step in the symplectic reduction procedure is to rewrite the problem on the quotient space of $\mu^{-1}(0)$ by an isotropy group, $G_\mu$. In the case of a group of symmetries forming a solvable Lie algebra, $G = G_0$, and the reduced symplectic manifold is $M_\mu = \mu^{-1}(0)/G$ with dimension $\dim M - 2 \dim G$. The reduced symplectic form is then defined as $i_\mu^*\omega = \pi_\mu^*\omega_\mu$ with $i_\mu : \mu^{-1}(0) \rightarrow M$ and $\pi_\mu : \mu^{-1}(0) \rightarrow M_\mu$. The inclusion map, $i_\mu$, is evaluated relatively easily knowing values for $\mu$, however, the canonical projection, $\pi_\mu$, is a principal $G$-bundle and it is typically not a trivial task to evaluate and herein lies a significant practical issue of applying the reduction process to general problems. \\

% \section{Expected Results and Analysis}
% \textit{Add discussion of the expected results, how they will be analyzed, etc.}

% \section{How to add Citations and a References List}

% You can upload a \verb|.bib| file containing your BibTeX entries, created with JabRef; or import your \href{https://www.overleaf.com/blog/184}{Mendeley}, CiteULike or Zotero library as a \verb|.bib| file. You can then cite entries from it, like this: \cite{greenwade93}. Just remember to specify a bibliography style, as well as the filename of the \verb|.bib|.

% You can find a \href{https://www.overleaf.com/help/97-how-to-include-a-bibliography-using-bibtex}{video tutorial here} to learn more about BibTeX.

% We hope you find Overleaf useful, and please let us know if you have any feedback using the help menu above --- or use the contact form at \url{https://www.overleaf.com/contact}!

% \bibliographystyle{alpha}
% \bibliography{sample}

\end{document}